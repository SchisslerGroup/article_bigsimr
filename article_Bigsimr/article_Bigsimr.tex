\documentclass[
]{jss}

\usepackage[utf8]{inputenc}

\providecommand{\tightlist}{%
  \setlength{\itemsep}{0pt}\setlength{\parskip}{0pt}}

\author{
Alfred G. Schissler\\University of Nevada \And Edward J.
Bedrick\\University of Arizona \And Alexander D. Knudson\\University of
Nevada \AND Tomasz J. Kozubowski\\University of Nevada \And Tin
Nguyen\\University of Nevada \And Anna K. Panorska\\University of Nevada
\AND Juli Petereit\\University of Nevada \And Walter W.
Piegorsch\\University of Arizona \And Duc Tran\\University of Nevada
}
\title{Simulating High-Dimensional Multivariate Data using the
\pkg{Bigsimr} Package}

\Plainauthor{Alfred G. Schissler, Edward J. Bedrick, Alexander D.
Knudson, Tomasz J. Kozubowski, Tin Nguyen, Anna K. Panorska, Juli
Petereit, Walter W. Piegorsch, Duc Tran}
\Plaintitle{Simulating High-Dimensional Multivariate Data using the
Bigsimr Package}
\Shorttitle{\pkg{Bigsimr}: Simulate High-Dimensional Multivariate Data}

\Abstract{
It is critical to accurately simulate data when employing Monte Carlo
techniques and evaluating statistical methodology. Measurements are
often correlated and high dimensional in this era of big data, such as
data obtained in high-throughput biomedical experiments. Due to the
computational complexity and a lack of user-friendly software available
to simulate these massive multivariate constructions, researchers resort
to simulation designs that posit independence or perform arbitrary data
transformations. To close this gap, we developed the \pkg{Bigsimr} Julia
package with R and Python interfaces. These packages empower
high-dimensional random vector simulation with arbitrary marginal
distributions and dependency via a Pearson, Spearman, or Kendall
correlation matrix. \pkg{Bigsimr} contains high-performance features,
including multi-core and graphical-processing-unit-accelerated
algorithms to estimate correlation and compute the nearest correlation
matrix. Monte Carlo studies quantify the accuracy and scalability of our
approach, up to \(d=10,000\). We describe example workflows and apply to
a high-dimensional data set --- RNA-sequencing data obtained from breast
cancer tumor samples.
}

\Keywords{multivariate simulation, high-dimensional data, nonparametric
correlation, Gaussian copula, RNA-sequencing data, breast cancer}
\Plainkeywords{multivariate simulation, high-dimensional
data, nonparametric correlation, Gaussian copula, RNA-sequencing
data, breast cancer}

%% publication information
%% \Volume{50}
%% \Issue{9}
%% \Month{June}
%% \Year{2012}
%% \Submitdate{}
%% \Acceptdate{2012-06-04}

\Address{
    Alfred G. Schissler\\
    University of Nevada\\
    1664 N Virginia St.\\
Reno, NV 89557\\
  E-mail: \email{aschissler@unr.edu}\\
  
                  }

% Pandoc citation processing

% Pandoc header

\usepackage{amsmath}

\begin{document}

\clearpage

\hypertarget{introduction}{%
\section{Introduction}\label{introduction}}

Massive high-dimensional (HD) data sets are now common in many areas of
scientific inquiry. As new methods are developed for data analysis, a
fundamental challenge lies in designing and conducting simulation
studies to assess the operating characteristics of proposed methodology
--- such as false positive rates, statistical power, interval coverage,
and robustness. Further, efficient simulation empowers statistical
computing strategies, such as the parametric bootstrap
\citep{Chernick2008} to simulate from a hypothesized null model,
providing inference in analytically challenging settings. Such Monte
Carlo (MC) techniques become difficult for HD dependent data using
existing algorithms and tools. This is particularly true when simulating
massive multivariate, non-normal distributions, arising in many fields
of study.

As others have noted, it can be vexing to simulate dependent,
non-normal/discrete data, even for low-dimensional (LD) settings
\citep{MB13, XZ19}. For continuous non-normal LD multivariate data, the
well-known NORmal To Anything (NORTA) algorithm \citep{Cario1997} and
other copula approaches \citep{Nelsen2007} are well-studied and
implemented in publicly-available software \citep{Yan2007, Chen2001}.
Yet these approaches do not scale in a timely fashion to HD problems
\citep{Li2019gpu}. For discrete data, early simulation strategies had
major flaws, such as failing to obtain the full range of possible
correlations --- such as admitting only positive correlations
\citep{Park1996}. While more recent approaches \citep{MB13, Xia17, BF17}
have remedied this issue for LD problems, the existing tools are not
designed to scale to high dimensions.

Another central issue lies in characterizing dependence between
components in the HD random vector. The choice of correlation in
practice usually relates to the eventual analytic goal and
distributional assumptions of the data (e.g., non-normal, discrete,
infinite support, etc.). For normal data, the Pearson product-moment
correlation describes the dependence perfectly. However, simulating
arbitrary random vectors that match a target Pearson correlation matrix
is computationally intense \citep{Chen2001, Xia17}. On the other hand,
an analyst might consider use of nonparametric correlation measures to
better characterize monotone, non-linear dependence, such as Spearman's
\(\rho\) and Kendall's \(\tau\). Throughout, we focus on matching these
nonparametric dependence measures, as our aim lies in modeling
non-normal data and these rank-based measures possess invariance
properties enabling our proposed methodology. We do, however, implement
Pearson matching, but several layers of approximation are required.

With all this in mind, we present a scalable, flexible multivariate
simulation algorithm. The crux of the method lies in the construction of
a Gaussian copula in the spirit of the NORTA procedure. As we will
describe in more detail, the algorithm's design leverages useful
properties of nonparametric correlation measures, namely invariance
under monotone transformation and well-known closed-form relationships
between dependence measures for the multivariate normal (MVN)
distribution. For our method, we developed a high-performance
implementation: the \texttt{Bigsimr} Julia package, with R and Python
interfaces \texttt{bigsimr}.

This article proceeds by providing background information, including a
description of a motivating example application: RNA-sequencing
(RNA-seq) breast cancer data. Then we describe and justify our
simulation methodology and related algorithms. We proceed by providing
an illustrative LD \texttt{bigsimr} workflow. Next we conduct MC studies
under various bivariate distributional assumptions to evaluate
performance and accuracy. After the MC evaluations, we simulate random
vectors motivated by our RNA-seq example, evaluate the accuracy, and
provide example statistical computing tasks, namely MC estimation of
joint probabilities and evaluating HD correlation estimation efficiency.
Finally, we discuss the method's utility, limitations, and future
directions.

\bibliography{bigsimr.bib,packages.bib,alex.bib}


\end{document}
